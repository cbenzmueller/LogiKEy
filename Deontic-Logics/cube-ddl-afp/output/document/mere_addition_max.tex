%
\begin{isabellebody}%
\setisabellecontext{mere{\isacharunderscore}{\kern0pt}addition{\isacharunderscore}{\kern0pt}max}%
%
\isadelimdocument
%
\endisadelimdocument
%
\isatagdocument
%
\isamarkupsection{Study Mere Addition Paradox: max rule%
}
\isamarkuptrue%
%
\endisatagdocument
{\isafolddocument}%
%
\isadelimdocument
%
\endisadelimdocument
%
\begin{isamarkuptext}%
This section studies the mere addition scenario \cite{Parfit1984-PARRAP,ddl:T87} when assuming the max rule.%
\end{isamarkuptext}\isamarkuptrue%
%
\isadelimtheory
%
\endisadelimtheory
%
\isatagtheory
\isacommand{theory}\isamarkupfalse%
\ mere{\isacharunderscore}{\kern0pt}addition{\isacharunderscore}{\kern0pt}max\ \ \isanewline
\isakeyword{imports}\ DDLcube\isanewline
\isanewline
\isakeyword{begin}%
\endisatagtheory
{\isafoldtheory}%
%
\isadelimtheory
%
\endisadelimtheory
\isanewline
\isanewline
\isacommand{consts}\isamarkupfalse%
\ A{\isacharcolon}{\kern0pt}{\isacharcolon}{\kern0pt}{\isasymsigma}\ Aplus{\isacharcolon}{\kern0pt}{\isacharcolon}{\kern0pt}{\isasymsigma}\ B{\isacharcolon}{\kern0pt}{\isacharcolon}{\kern0pt}{\isasymsigma}\ \ i{\isadigit{1}}{\isacharcolon}{\kern0pt}{\isacharcolon}{\kern0pt}i\ i{\isadigit{2}}{\isacharcolon}{\kern0pt}{\isacharcolon}{\kern0pt}i\ i{\isadigit{3}}{\isacharcolon}{\kern0pt}{\isacharcolon}{\kern0pt}i\ i{\isadigit{4}}{\isacharcolon}{\kern0pt}{\isacharcolon}{\kern0pt}i\ i{\isadigit{5}}{\isacharcolon}{\kern0pt}{\isacharcolon}{\kern0pt}i\ i{\isadigit{6}}{\isacharcolon}{\kern0pt}{\isacharcolon}{\kern0pt}i\ i{\isadigit{7}}{\isacharcolon}{\kern0pt}{\isacharcolon}{\kern0pt}i\ i{\isadigit{8}}{\isacharcolon}{\kern0pt}{\isacharcolon}{\kern0pt}i\ \isanewline
\isanewline
\isacommand{axiomatization}\isamarkupfalse%
\ \isakeyword{where}\isanewline
\ %
\isamarkupcmt{A is striclty better than B%
}\isanewline
\ PP{\isadigit{0}}{\isacharcolon}{\kern0pt}\ {\isachardoublequoteopen}{\isasymlfloor}{\isacharparenleft}{\kern0pt}\isactrlbold {\isasymnot}{\isasymcircle}{\isacharless}{\kern0pt}\isactrlbold {\isasymnot}A{\isacharbar}{\kern0pt}A\isactrlbold {\isasymor}B{\isachargreater}{\kern0pt}\isactrlbold {\isasymand}{\isasymcircle}{\isacharless}{\kern0pt}\isactrlbold {\isasymnot}B{\isacharbar}{\kern0pt}A\isactrlbold {\isasymor}B{\isachargreater}{\kern0pt}{\isacharparenright}{\kern0pt}{\isasymrfloor}{\isachardoublequoteclose}\ \isakeyword{and}\isanewline
\ %
\isamarkupcmt{Aplus is at least as good as A%
}\isanewline
\ PP{\isadigit{1}}{\isacharcolon}{\kern0pt}\ {\isachardoublequoteopen}{\isasymlfloor}\isactrlbold {\isasymnot}{\isasymcircle}{\isacharless}{\kern0pt}\isactrlbold {\isasymnot}Aplus{\isacharbar}{\kern0pt}A\isactrlbold {\isasymor}Aplus{\isachargreater}{\kern0pt}{\isasymrfloor}{\isachardoublequoteclose}\ \isakeyword{and}\isanewline
\ %
\isamarkupcmt{B is strictly better than Aplus%
}\isanewline
\ PP{\isadigit{2}}{\isacharcolon}{\kern0pt}\ {\isachardoublequoteopen}{\isasymlfloor}{\isacharparenleft}{\kern0pt}\isactrlbold {\isasymnot}{\isasymcircle}{\isacharless}{\kern0pt}\isactrlbold {\isasymnot}B{\isacharbar}{\kern0pt}Aplus\isactrlbold {\isasymor}B{\isachargreater}{\kern0pt}\ \isactrlbold {\isasymand}\ {\isasymcircle}{\isacharless}{\kern0pt}\isactrlbold {\isasymnot}Aplus{\isacharbar}{\kern0pt}Aplus\isactrlbold {\isasymor}B{\isachargreater}{\kern0pt}{\isacharparenright}{\kern0pt}{\isasymrfloor}{\isachardoublequoteclose}%
\begin{isamarkuptext}%
Nitpick finds no finite model for the betterness 
   relation.%
\end{isamarkuptext}\isamarkuptrue%
\isacommand{theorem}\isamarkupfalse%
\ T{\isadigit{0}}{\isacharcolon}{\kern0pt}\isanewline
\ \ \isakeyword{assumes}\ transitivity\ \ \isanewline
\ \ \isakeyword{shows}\ True\isanewline
\ \ \isacommand{nitpick}\isamarkupfalse%
\ {\isacharbrackleft}{\kern0pt}satisfy{\isacharbrackright}{\kern0pt}\ %
\isamarkupcmt{no model found%
}\isanewline
%
\isadelimproof
\ \ %
\endisadelimproof
%
\isatagproof
\isacommand{oops}\isamarkupfalse%
%
\endisatagproof
{\isafoldproof}%
%
\isadelimproof
%
\endisadelimproof
%
\begin{isamarkuptext}%
Nitpick shows consistency in the absence of transitivity.%
\end{isamarkuptext}\isamarkuptrue%
\isacommand{theorem}\isamarkupfalse%
\ T{\isadigit{1}}{\isacharcolon}{\kern0pt}\isanewline
\ \ \isakeyword{shows}\ True\isanewline
\ \ \isacommand{nitpick}\isamarkupfalse%
\ {\isacharbrackleft}{\kern0pt}satisfy{\isacharcomma}{\kern0pt}expect{\isacharequal}{\kern0pt}genuine{\isacharcomma}{\kern0pt}card\ i{\isacharequal}{\kern0pt}{\isadigit{3}}{\isacharbrackright}{\kern0pt}\ \ %
\isamarkupcmt{model found%
}\isanewline
%
\isadelimproof
\ \ %
\endisadelimproof
%
\isatagproof
\isacommand{oops}\isamarkupfalse%
%
\endisatagproof
{\isafoldproof}%
%
\isadelimproof
%
\endisadelimproof
%
\begin{isamarkuptext}%
Sledgehammer confirms inconsistency in the presence of the interval order condition.%
\end{isamarkuptext}\isamarkuptrue%
\isacommand{theorem}\isamarkupfalse%
\ T{\isadigit{2}}{\isacharcolon}{\kern0pt}\isanewline
\ \ \isakeyword{assumes}\ reflexivity\ \isakeyword{and}\ Ferrers\isanewline
\ \ \isakeyword{shows}\ False\isanewline
\ \ %
\isamarkupcmt{sledgehammer%
}\isanewline
%
\isadelimproof
\ \ %
\endisadelimproof
%
\isatagproof
\isacommand{by}\isamarkupfalse%
\ {\isacharparenleft}{\kern0pt}metis\ PP{\isadigit{0}}\ PP{\isadigit{1}}\ PP{\isadigit{2}}\ assms{\isacharparenleft}{\kern0pt}{\isadigit{1}}{\isacharparenright}{\kern0pt}\ assms{\isacharparenleft}{\kern0pt}{\isadigit{2}}{\isacharparenright}{\kern0pt}{\isacharparenright}{\kern0pt}%
\endisatagproof
{\isafoldproof}%
%
\isadelimproof
%
\endisadelimproof
%
\begin{isamarkuptext}%
Nitpick shows consistency if transitivity is weakened into acyclicity.%
\end{isamarkuptext}\isamarkuptrue%
\isacommand{theorem}\isamarkupfalse%
\ T{\isadigit{3}}{\isacharcolon}{\kern0pt}\isanewline
\ \ \isakeyword{assumes}\ loopfree\isanewline
\ \ \isakeyword{shows}\ True\isanewline
\ \ \isacommand{nitpick}\isamarkupfalse%
\ {\isacharbrackleft}{\kern0pt}satisfy{\isacharcomma}{\kern0pt}expect{\isacharequal}{\kern0pt}genuine{\isacharcomma}{\kern0pt}card{\isacharequal}{\kern0pt}{\isadigit{3}}{\isacharbrackright}{\kern0pt}\ %
\isamarkupcmt{model found%
}\isanewline
%
\isadelimproof
\ \ %
\endisadelimproof
%
\isatagproof
\isacommand{oops}\isamarkupfalse%
%
\endisatagproof
{\isafoldproof}%
%
\isadelimproof
%
\endisadelimproof
%
\begin{isamarkuptext}%
Transitivity or quasi-transitivity: Nitpick shows inconsistency assuming a finite model
   of cardinality (up to) seven (if we provide the exact dependencies); for higher cardinalities 
   it returns a time out (depending on computing it may prove falsity also for cardinality eight, 
   etc.%
\end{isamarkuptext}\isamarkuptrue%
\isacommand{theorem}\isamarkupfalse%
\ T{\isadigit{4}}{\isacharcolon}{\kern0pt}\isanewline
\ \ \ \ \isakeyword{assumes}\isanewline
\ \ \ \ \ \ transitivity\ \isakeyword{and}\isanewline
\ \ \ \ \ \ OnlyOnes{\isacharcolon}{\kern0pt}\ {\isachardoublequoteopen}{\isasymforall}y{\isachardot}{\kern0pt}\ y{\isacharequal}{\kern0pt}i{\isadigit{1}}\ {\isasymor}\ y{\isacharequal}{\kern0pt}i{\isadigit{2}}\ {\isasymor}\ y{\isacharequal}{\kern0pt}i{\isadigit{3}}\ {\isasymor}\ y{\isacharequal}{\kern0pt}i{\isadigit{4}}\ {\isasymor}\ y{\isacharequal}{\kern0pt}\ i{\isadigit{5}}\ {\isasymor}\ y{\isacharequal}{\kern0pt}\ i{\isadigit{6}}\ {\isasymor}\ y{\isacharequal}{\kern0pt}\ i{\isadigit{7}}{\isachardoublequoteclose}\isanewline
\ \ \ \ \isakeyword{shows}\ False\isanewline
%
\isadelimproof
\ \ \ \ %
\endisadelimproof
%
\isatagproof
\isacommand{using}\isamarkupfalse%
\ assfactor{\isacharunderscore}{\kern0pt}def\ PP{\isadigit{0}}\ PP{\isadigit{1}}\ PP{\isadigit{2}}\ assms\ \isanewline
\ \ %
\isamarkupcmt{sledgehammer()%
}\ \isanewline
\ \ %
\isamarkupcmt{proof found by Sledgehammer, but reconstruction fails%
}\isanewline
\ \ \isacommand{oops}\isamarkupfalse%
%
\endisatagproof
{\isafoldproof}%
%
\isadelimproof
\isanewline
%
\endisadelimproof
\isanewline
\isacommand{theorem}\isamarkupfalse%
\ T{\isadigit{5}}{\isacharcolon}{\kern0pt}\isanewline
\ \ \ \ \isakeyword{assumes}\isanewline
\ \ \ \ \ \ Quasitransit\ \isakeyword{and}\isanewline
\ \ \ \ \ \ OnlyOnes{\isacharcolon}{\kern0pt}\ {\isachardoublequoteopen}{\isasymforall}y{\isachardot}{\kern0pt}\ y{\isacharequal}{\kern0pt}i{\isadigit{1}}\ {\isasymor}\ y{\isacharequal}{\kern0pt}i{\isadigit{2}}\ {\isasymor}\ y{\isacharequal}{\kern0pt}i{\isadigit{3}}\ {\isasymor}\ y{\isacharequal}{\kern0pt}i{\isadigit{4}}\ {\isasymor}\ y{\isacharequal}{\kern0pt}\ i{\isadigit{5}}\ {\isasymor}\ y{\isacharequal}{\kern0pt}\ i{\isadigit{6}}\ {\isasymor}\ y{\isacharequal}{\kern0pt}i{\isadigit{7}}{\isachardoublequoteclose}\isanewline
\ \ \ \ \isakeyword{shows}\ False\isanewline
%
\isadelimproof
\ \ %
\endisadelimproof
%
\isatagproof
\isacommand{using}\isamarkupfalse%
\ assfactor{\isacharunderscore}{\kern0pt}def\ PP{\isadigit{0}}\ PP{\isadigit{1}}\ PP{\isadigit{2}}\ assms\isanewline
\ \ %
\isamarkupcmt{sledgehammer()%
}\isanewline
\ \ %
\isamarkupcmt{proof found by Sledgehammer, but reconstruction fails%
}\isanewline
\ \ \isacommand{oops}\isamarkupfalse%
%
\endisatagproof
{\isafoldproof}%
%
\isadelimproof
%
\endisadelimproof
%
\begin{isamarkuptext}%
Testing whether infinity holds — infinity is defined as: there is a surjective mapping G from 
   domain i to a proper subset M of domain i.%
\end{isamarkuptext}\isamarkuptrue%
\isacommand{abbreviation}\isamarkupfalse%
\ {\isachardoublequoteopen}infinity\ {\isasymequiv}\ {\isasymexists}M{\isachardot}{\kern0pt}\ {\isacharparenleft}{\kern0pt}{\isasymexists}z{\isacharcolon}{\kern0pt}{\isacharcolon}{\kern0pt}i{\isachardot}{\kern0pt}\ {\isasymnot}{\isacharparenleft}{\kern0pt}M\ z{\isacharparenright}{\kern0pt}\ {\isasymand}\ {\isacharparenleft}{\kern0pt}{\isasymexists}G{\isachardot}{\kern0pt}\ {\isacharparenleft}{\kern0pt}{\isasymforall}y{\isacharcolon}{\kern0pt}{\isacharcolon}{\kern0pt}i{\isachardot}{\kern0pt}\ {\isacharparenleft}{\kern0pt}{\isasymexists}x{\isachardot}{\kern0pt}\ {\isacharparenleft}{\kern0pt}M\ x{\isacharparenright}{\kern0pt}\ {\isasymand}\ {\isacharparenleft}{\kern0pt}G\ x{\isacharparenright}{\kern0pt}\ {\isacharequal}{\kern0pt}\ y{\isacharparenright}{\kern0pt}{\isacharparenright}{\kern0pt}{\isacharparenright}{\kern0pt}{\isacharparenright}{\kern0pt}{\isachardoublequoteclose}\isanewline
\isanewline
\isacommand{lemma}\isamarkupfalse%
\ {\isachardoublequoteopen}infinity{\isachardoublequoteclose}\ \isacommand{nitpick}\isamarkupfalse%
{\isacharbrackleft}{\kern0pt}expect{\isacharequal}{\kern0pt}genuine{\isacharbrackright}{\kern0pt}%
\isadelimproof
\ %
\endisadelimproof
%
\isatagproof
\isacommand{oops}\isamarkupfalse%
\ %
\isamarkupcmt{countermodel found%
}%
\endisatagproof
{\isafoldproof}%
%
\isadelimproof
%
\endisadelimproof
%
\begin{isamarkuptext}%
Now we study infinity under the assumption of (quasi-)transitivity: we do 
not get any finite countermodels reported anymore.%
\end{isamarkuptext}\isamarkuptrue%
\isacommand{lemma}\isamarkupfalse%
\ \isanewline
\ \ \isakeyword{assumes}\ transitivity\isanewline
\ \ \isakeyword{shows}\ \ \ infinity\isanewline
\ \ %
\isamarkupcmt{nitpick%
}\ \ \ %
\isamarkupcmt{no countermodel found anymore; nitpicks runs out of time%
}\isanewline
\ \ %
\isamarkupcmt{sledgehammer%
}\ \ %
\isamarkupcmt{but the provers are still too weak to prove it automatically%
}\isanewline
%
\isadelimproof
\ \ %
\endisadelimproof
%
\isatagproof
\isacommand{oops}\isamarkupfalse%
%
\endisatagproof
{\isafoldproof}%
%
\isadelimproof
\isanewline
%
\endisadelimproof
\isanewline
\isacommand{lemma}\isamarkupfalse%
\ \isanewline
\ \ \isakeyword{assumes}\ Quasitransit\ \isanewline
\ \ \isakeyword{shows}\ \ \ infinity\isanewline
\ \ %
\isamarkupcmt{nitpick%
}\ \ \ %
\isamarkupcmt{no countermodel found anymore; nitpicks runs out of time%
}\isanewline
\ \ %
\isamarkupcmt{sledgehammer%
}\ \ %
\isamarkupcmt{but the provers are still too weak to prove it automatically%
}\isanewline
%
\isadelimproof
\ \ \ %
\endisadelimproof
%
\isatagproof
\isacommand{oops}\isamarkupfalse%
%
\endisatagproof
{\isafoldproof}%
%
\isadelimproof
\isanewline
%
\endisadelimproof
\isanewline
\isacommand{theorem}\isamarkupfalse%
\ T{\isadigit{0}}{\isacharprime}{\kern0pt}{\isacharcolon}{\kern0pt}\isanewline
\ \ \isakeyword{assumes}\ transitivity\ \isakeyword{and}\ totality\ \ \isanewline
\ \ \isakeyword{shows}\ False\isanewline
\ \ %
\isamarkupcmt{sledgehammer%
}\isanewline
%
\isadelimproof
\ \ %
\endisadelimproof
%
\isatagproof
\isacommand{by}\isamarkupfalse%
\ {\isacharparenleft}{\kern0pt}metis\ PP{\isadigit{0}}\ PP{\isadigit{1}}\ PP{\isadigit{2}}\ assms{\isacharparenleft}{\kern0pt}{\isadigit{1}}{\isacharparenright}{\kern0pt}\ assms{\isacharparenleft}{\kern0pt}{\isadigit{2}}{\isacharparenright}{\kern0pt}{\isacharparenright}{\kern0pt}%
\endisatagproof
{\isafoldproof}%
%
\isadelimproof
\ \isanewline
%
\endisadelimproof
%
\isadelimtheory
\ \ \isanewline
%
\endisadelimtheory
%
\isatagtheory
\isacommand{end}\isamarkupfalse%
%
\endisatagtheory
{\isafoldtheory}%
%
\isadelimtheory
%
\endisadelimtheory
%
\end{isabellebody}%
\endinput
%:%file=~/GITHUBS/LogiKEy/Deontic-Logics/cube-ddl-afp/mere_addition_max.thy%:%
%:%11=1%:%
%:%23=3%:%
%:%31=5%:%
%:%32=5%:%
%:%33=6%:%
%:%34=7%:%
%:%35=8%:%
%:%42=8%:%
%:%43=9%:%
%:%44=10%:%
%:%45=10%:%
%:%46=11%:%
%:%47=12%:%
%:%48=12%:%
%:%49=13%:%
%:%50=13%:%
%:%51=13%:%
%:%52=14%:%
%:%53=15%:%
%:%54=15%:%
%:%55=15%:%
%:%56=16%:%
%:%57=17%:%
%:%58=17%:%
%:%59=17%:%
%:%60=18%:%
%:%62=21%:%
%:%63=22%:%
%:%65=24%:%
%:%66=24%:%
%:%67=25%:%
%:%68=26%:%
%:%69=27%:%
%:%70=27%:%
%:%71=27%:%
%:%72=27%:%
%:%75=28%:%
%:%79=28%:%
%:%89=30%:%
%:%91=32%:%
%:%92=32%:%
%:%93=33%:%
%:%94=34%:%
%:%95=34%:%
%:%96=34%:%
%:%97=34%:%
%:%100=35%:%
%:%104=35%:%
%:%114=37%:%
%:%116=39%:%
%:%117=39%:%
%:%118=40%:%
%:%119=41%:%
%:%120=42%:%
%:%121=42%:%
%:%122=42%:%
%:%125=43%:%
%:%129=43%:%
%:%130=43%:%
%:%139=45%:%
%:%141=47%:%
%:%142=47%:%
%:%143=48%:%
%:%144=49%:%
%:%145=50%:%
%:%146=50%:%
%:%147=50%:%
%:%148=50%:%
%:%151=51%:%
%:%155=51%:%
%:%165=53%:%
%:%166=54%:%
%:%167=55%:%
%:%168=56%:%
%:%170=58%:%
%:%171=58%:%
%:%172=59%:%
%:%173=60%:%
%:%174=61%:%
%:%175=62%:%
%:%178=63%:%
%:%182=63%:%
%:%183=63%:%
%:%184=64%:%
%:%185=64%:%
%:%186=64%:%
%:%187=65%:%
%:%188=65%:%
%:%189=65%:%
%:%190=66%:%
%:%196=66%:%
%:%199=67%:%
%:%200=68%:%
%:%201=68%:%
%:%202=69%:%
%:%203=70%:%
%:%204=71%:%
%:%205=72%:%
%:%208=73%:%
%:%212=73%:%
%:%213=73%:%
%:%214=74%:%
%:%215=74%:%
%:%216=74%:%
%:%217=75%:%
%:%218=75%:%
%:%219=75%:%
%:%220=76%:%
%:%230=78%:%
%:%231=79%:%
%:%233=81%:%
%:%234=81%:%
%:%235=82%:%
%:%236=83%:%
%:%237=83%:%
%:%238=83%:%
%:%240=83%:%
%:%244=83%:%
%:%245=83%:%
%:%246=83%:%
%:%256=85%:%
%:%257=86%:%
%:%259=88%:%
%:%260=88%:%
%:%261=89%:%
%:%262=90%:%
%:%263=91%:%
%:%264=91%:%
%:%265=91%:%
%:%266=91%:%
%:%267=91%:%
%:%268=92%:%
%:%269=92%:%
%:%270=92%:%
%:%271=92%:%
%:%272=92%:%
%:%275=93%:%
%:%279=93%:%
%:%285=93%:%
%:%288=94%:%
%:%289=95%:%
%:%290=95%:%
%:%291=96%:%
%:%292=97%:%
%:%293=98%:%
%:%294=98%:%
%:%295=98%:%
%:%296=98%:%
%:%297=98%:%
%:%298=99%:%
%:%299=99%:%
%:%300=99%:%
%:%301=99%:%
%:%302=99%:%
%:%305=100%:%
%:%309=100%:%
%:%315=100%:%
%:%318=101%:%
%:%319=102%:%
%:%320=102%:%
%:%321=103%:%
%:%322=104%:%
%:%323=105%:%
%:%324=105%:%
%:%325=105%:%
%:%328=106%:%
%:%332=106%:%
%:%333=106%:%
%:%338=106%:%
%:%343=107%:%
%:%348=108%:%
